\section{Introduction}

Parkinson's disease (PD) is a progressive neurodegenerative disorder affecting
over 8.5 million people globally as of 2019 \cite{WHO2022}, with cases rising
due to aging populations and improved diagnostics. A significant challenge for
about 90\% of individuals with PD is speech impairment \cite{IndianaParkinson},
known as hypokinetic dysarthria, which includes reduced vocal loudness,
monotone speech, imprecise articulation, and variable speech rates. Early
detection of PD through the analysis of speech can confer immense benefit,
including earlier access to traditional and experimental treatment, which can,
in turn, help to delay the progression of the disease.

Current methods for detecting PD using speech features include:

\begin{itemize}
	\item \textbf{Support Vector Machines (SVM):} Used for classifying PD
	      patients by analyzing speech features like Mel Frequency Ceptral
	      Coefficients (MFCC) and AutoEncoder patterns
	      \cite{SpringerSpeechProcessing}.
	\item \textbf{Deep Neural Networks (DNN):} Employed to model complex patterns
	      in speech data for accurate PD detection \cite{AIPDNN}.
	\item \textbf{Random Forests (RF):} Utilized to identify important speech
	      features and classify patients based on PD-related symptoms
	      \cite{SpringerMLPD}.
\end{itemize}

While these methods demonstrate considerable success, this project aims to
explore the effectiveness of K-Nearest Neighbors (KNN) for PD detection.
Specifically, three KNN models will be developed: one trained on cleaned data,
another on aggregated data, and a third on normalized data. KNN is a simple,
non-parametric algorithm that requires minimal assumptions about the data
distribution, making it well-suited for small datasets and multi-class
problems. Additionally, an API will be created to enable users to test the
models by providing their own input values, offering an interactive tool for PD
prediction.

\cref{fig:fig1} illustrates project timeline and the task distribution
spanning from 8/01/2025 to 19/01/2025.

\usetikzlibrary{arrows, positioning, shapes.geometric,
	decorations.pathreplacing}

\begin{figure}[H]
	\centering
	\begin{adjustbox}{scale=0.80}
		\begin{ganttchart}[
				vgrid,
				hgrid,
				x unit = 0.7cm,
				y unit title = 0.6cm,
				y unit chart = 0.6cm,
				title/.append style = {fill = gray!40},
				title height = 1,
				bar height = 0.4,
				bar/.append style = {fill = green!90, rounded corners = 3pt},
				bar incomplete/.append style = {fill = gray!50},
				bar label node/.append style = {align = left, text width = 12em},
				bar label font = \footnotesize,
				progress label text = {},
				milestone label font = \bfseries\small,
				milestone label node/.append style = {align = left, text width = 12em},
				milestone height = 0.8,
				milestone top shift = 0.2,
				milestone/.append style = {fill = red, xscale=0.85},
				group/.append style = {draw = black, fill = yellow!90},
				group label node/.append style = {align = left, text width = 12em},
				group height = .3,
				group peaks height = .2,
				group label font = \bfseries\small,
				group left shift = 0,
				group right shift = 0,
				group top shift = .3,
				group height = .2,
			]{1}{12}

			\gantttitle[
				title label text = {Day},
				title label font = \bfseries\footnotesize
			]{8}{12} \\
			\gantttitlelist[title label font =
				\tiny]{8,9,10,11,12,13,14,15,16,17,18,19}{1} \\

			\ganttgroup{Processing}{1}{5} \\
			\ganttbar{Renaming (Scarlett)}{1}{1} \\
			\ganttbar{Outlier Correction (Scarlett)}{2}{2} \\
			\ganttbar{Aggregating (Roger)}{3}{3} \\
			\ganttbar{Normalizing (Rahul)}{4}{5} \\

			\ganttgroup{Model}{6}{7} \\
			\ganttbar{Collinearity (Roger)}{6}{6} \\
			\ganttbar{Training (Annabel)}{7}{7} \\

			\ganttgroup{API}{8}{9} \\
			\ganttbar{Front-end (Rahul/Annabel)}{8}{8} \\
			\ganttbar{Back-end (Rahul/Annabel)}{9}{9} \\

			\ganttgroup{Report and Video}{10}{12} \\
			\ganttbar{Writing Report (All)}{10}{11} \\
			\ganttbar{Recording Video (Roger)}{12}{12} \\

			\ganttmilestone{Completion}{12}
		\end{ganttchart}
	\end{adjustbox}
	\caption{Gantt chart of the project.}
	\label{fig:fig1}
\end{figure}
